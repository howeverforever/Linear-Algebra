% === Exercise 7.2.4 ===
\begin{Exercise}
	\begin{enumerate}[(a)]
		\item
		\begin{answer}
			$\mathcal{J} = \begin{pmatrix}
			1 & 0 & 0 \\
			0 & 2 & 1 \\
			0 & 0 & 2
			\end{pmatrix}$. $Q = \begin{pmatrix}
			1 & 1 & 1 \\
			2 & 1 & 2 \\
			1 & -1 & 0
			\end{pmatrix}$. 
		\end{answer}
		\begin{solution}
			Firstly, we need to find eigenvalues. Solve $\det(A-\lambda I) = -(\lambda-1)(\lambda-2)^2 = 0$. Then $\lambda_1 = 1$, $\lambda_2 = 2$.
			
			Secondly, we need to find a basis. For $\lambda_1 = 1$, since it is with multiplicity $1$, then we just pick eigenvector $\begin{pmatrix}
			1 \\
			2 \\
			1
			\end{pmatrix}$ to be a basis. For $\lambda_2 = 2$, we know $T-2I = \begin{pmatrix}
			-5 & 3 & -2 \\
			-7 & 4 & -3 \\
			1 & -1 & 0
			\end{pmatrix}$ and $(T-2I)^2 = \begin{pmatrix}
			2 & -1 & 1 \\
			4 & -2 & 2 \\
			2 & -1 & 0
			\end{pmatrix}$. Since $\lambda_2$ is with multiplicity $2$ and
			\begin{align*}
			&\dim(A)-\rank(T-2I) = 3-2=1; \\
			&\rank(T-2I)-\rank((T-2I)^2) = 2-1 = 1,
			\end{align*}
			we should find a vector $v$ such that $v\in N\big((T-2I)^2\big)$ and $v\notin N\big(T-2I\big)$. Here we pick $v = \begin{pmatrix}
			1 \\
			2 \\
			0
			\end{pmatrix}$, then $(T-2I)(v) = \begin{pmatrix}
			1 \\
			1 \\
			-1
			\end{pmatrix}$. 
			
			Hence this means $Q = \begin{pmatrix}
			1 & 1 & 1 \\
			2 & 1 & 2 \\
			1 & -1 & 0
			\end{pmatrix}$. $\mathcal{J} = Q^{-1} A Q = \begin{pmatrix}
			1 & 0 & 0 \\
			0 & 2 & 1 \\
			0 & 0 & 2
			\end{pmatrix}$ follows.
		\end{solution}
	\end{enumerate}
\end{Exercise}