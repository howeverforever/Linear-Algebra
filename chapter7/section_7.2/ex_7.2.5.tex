% === Exercise 7.2.5 ===
\begin{Exercise}
	Notice that a Jordan canonical basis $\beta$ is not unique.
	\begin{enumerate}[(a)]
		\item
		\begin{answer}
			$\beta = \{e^t, t e^t, t^2 e^t, e^{2t}\}$. $\mathcal{J} = \begin{pmatrix}
			1 & 1 & 0 & 0 \\
			0 & 1 & 1 & 0 \\
			0 & 0 & 1 & 0 \\
			0 & 0 & 0 & 2
			\end{pmatrix}$. 
		\end{answer}
		\begin{solution}
			Observe
			\begin{align*}
			T(e^t) &= e^t; \\
			T(t e^t) &= e^t + t e^t; \\
			T(t^2 e^t) &= 2t e^t + t^2 e^t; \\
			T(e^{2t}) &= 2e^{2t}.
			\end{align*}
			Pick $\beta = \{e^t, t e^t, t^2 e^t, e^{2t}\}$ which is a basis for $V$. Then we know
			$$
			[T]_{\beta} = \begin{pmatrix}
			1 & 0 & 0 & 0 \\
			1 & 1 & 0 & 0 \\
			0 & 2 & 1 & 0 \\
			0 & 0 & 0 & 2
			\end{pmatrix}.
			$$
			Solve $\det(T-\lambda I) = (\lambda-1)^3 (\lambda-2) = 0$, we obtain $\lambda_1 = 1$, $\lambda_2 = 2$ with multiplicities $3$,$1$, respectively.
			
			For $\lambda_1 = 1$, we know
			$$
			T-I = \begin{pmatrix}
			0 & 0 & 0 & 0 \\
			1 & 0 & 0 & 0 \\
			0 & 2 & 0 & 0 \\
			0 & 0 & 0 & 1
			\end{pmatrix};\
			(T-I)^2 = \begin{pmatrix}
			0 & 0 & 0 & 0 \\
			0 & 0 & 0 & 0 \\
			2 & 0 & 0 & 0 \\
			0 & 0 & 0 & 1
			\end{pmatrix}; 
			(T-I)^3 = \begin{pmatrix}
			0 & 0 & 0 & 0 \\
			0 & 0 & 0 & 0 \\
			0 & 0 & 0 & 0 \\
			0 & 0 & 0 & 1
			\end{pmatrix}.
			$$
			This implies
			\begin{align*}
			&\dim(T)-\rank(T-I) = 4-3 = 1; \\
			&\rank(T-I) - \rank\big((T-I)^2\big) = 3 - 1 = 2; \\
			&\rank\big((T-I)^2\big) - \rank\big((T-I)^3\big) = 2-1 = 1.
			\end{align*}
			We should find a vector $v$ such that $v\in N\big((T-I)^3\big)$, $v\notin N\big((T-I)^k\big)$ for $k=1,2$. Hence we pick $v = \begin{pmatrix}
			1 \\
			0 \\
			0 \\
			0
			\end{pmatrix}$. By doing so, $(T-I)(v) = \begin{pmatrix}
			0 \\
			1 \\
			0 \\
			0
			\end{pmatrix}$ and $(T-I)^2(v) = \begin{pmatrix}
			0 \\
			0 \\
			2 \\
			0
			\end{pmatrix}$ follow.
			For $\lambda_2 = 2$, since it has multiplicity $1$, we just consider $T-2I = \begin{pmatrix}
			-1 & 0 & 0 & 0 \\
			1 & -1 & 0 & 0 \\
			0 & 2 & -1 & 0 \\
			0 & 0 & 0 & 0
			\end{pmatrix}$, then pick an eigenvector $\begin{pmatrix}
			0 \\
			0 \\
			0 \\
			1
			\end{pmatrix}$.
			
			Hence we have $Q = \begin{pmatrix}
			0 & 0 & 1 & 0 \\
			0 & 1 & 0 & 0 \\
			2 & 0 & 0 & 0 \\
			0 & 0 & 0 & 1
			\end{pmatrix}$ such that $\mathcal{J} = Q^{-1} [T]_{\beta} Q$, then the Jordan canonical form $\mathcal{J} = \begin{pmatrix}
			1 & 1 & 0 & 0 \\
			0 & 1 & 1 & 0 \\
			0 & 0 & 1 & 0 \\
			0 & 0 & 0 & 2
			\end{pmatrix}$ follows.
			
			
		\end{solution}
	\end{enumerate}
\end{Exercise}