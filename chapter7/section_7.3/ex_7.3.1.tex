% === Exercise 7.3.1 ===
\begin{Exercise}
\begin{enumerate}[(a)]
\item[(a)]
\begin{answer}
False.
\end{answer}
\begin{solution}
Consider $T = \begin{pmatrix}
3 & -1 & 0 \\
0 & 2 & 0 \\
1 & -1 & 2
\end{pmatrix}$. We know the characteristic polynomial $f(t) = -(t-2)^2 (t-3)$. 
However, we can pick $p(t) = (t-2)(t-3)$ such that $p(T) = T_0$ where the degree of $p(t)$ is less than $f(t)$.
\end{solution}

\item[(b)]
\begin{answer}
True.
\end{answer}
\begin{solution}
It follows from Theorem 7.12(b).
\end{solution}

\item[(c)]
\begin{answer}
False.
\end{answer}
\begin{solution}
It should be that the minimal polynomial divides the characteristic polynomial. This follows from Theorem 7.12(a).
\end{solution}

\item[(d)]
\begin{answer}
False.
\end{answer}
\begin{solution}
Consider the identity linear operator $T$ which is diagonalizable. Its characteristic polynomial $(t-1)^2$; however its minimal polynomial is $t-1$. 
Hence they are different.
\end{solution}

\item[(e)]
\begin{answer}
True.
\end{answer}
\begin{solution}
By the Corollary of Theorem 7.14, we suppose 
$$
f(t) = (\lambda_1-t)^{n_1} (\lambda_2-t)^{n_2} \cdots (\lambda_k-t)^{n_k}
$$
where $\lambda_1, \lambda_2, \cdots, \lambda_k$ are different eigenvalues of $T$. Then there exists $m_1, m_2, \cdots, m_k$ where $1\leq m_i \leq n_i$ for each $i$ such that
$$
p(t) = (t-\lambda_1)^{m_1} (t-\lambda_2)^{m_2} \cdots (t-\lambda_k)^{m_k}.
$$
Notice that $n_1+n_2+\cdots+n_k = n$. 
Hence $f(t)$ divides $ \big[p(t)\big]^n$.
\end{solution}

\item[(f)]
\begin{answer}
False.
\end{answer}
\begin{solution}
The counter-example could see part (d).
\end{solution}

\item[(g)]
\begin{answer}
False.
\end{answer}
\begin{solution}
Consider $T = \begin{pmatrix}
1 & 1 \\
0 & 1
\end{pmatrix}$. We know its minimal polynomial $(t-1)^2$ could split.
However, $T$ is not diagonalizable.
\end{solution}

\item[(h)]
\begin{answer}
True.
\end{answer}
\begin{solution}
It follows from Theorem 7.15.
\end{solution}

\item[(i)]
\begin{answer}
True.
\end{answer}
\begin{solution}
We know the degree of the minimal polynomial must be greater than or equal to $n$ by the Corollary of Theorem 7.14.
By Cayley-Hamilton Theorem, its degree must be less than or equal to $n$.
Hence the degree of the minimal polynomial must be equal to $n$.
\end{solution}

\end{enumerate}
\end{Exercise}