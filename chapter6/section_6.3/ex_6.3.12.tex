% === Exercise 6.3.12 ===
\begin{Exercise}
	\begin{enumerate}[(a)]
		\item
		\begin{proof}
			It suffice to prove $R(T^*)^{\perp} \subseteq N(T)$ and $N(T) \subseteq R(T^*)^{\perp}$.
			
			First, we claim $N(T) \subseteq R(T^*)^{\perp}$. Let $x\in N(T)$, then $T x = 0$. For any $y\in V$, we consider $
			0 = \langle 0, y \rangle = \langle T x , y \rangle = \langle x, T^* y \rangle$. Hence $x\in R(T^*)^{\perp}$. Because $x$ was arbitrary, the claim holds.
			
			Secondly, we claim $R(T^*)^{\perp} \subseteq N(T)$. Let $x\in R(T^*)^{\perp}$, then for any $y\in V$, we consider $0 = \langle x, T^* y \rangle = \langle T x, y \rangle$. This implies $T x = 0$ and hence $x\in N(T)$. Because $x$ was arbitrary, the claim holds.
			
			Finally, we conclude $R(T^*)^{\perp} = N(T)$.
		\end{proof}
		
		\item
		\begin{proof}
			Since $V$ is finite-dimensional, then both $R(T^*)$ and $N(T)$ are finite-dimensional. From part (a), $(R(T^*)^{\perp})^{\perp} = N(T)^{\perp}$ follows. Hence we obtain $R(T^*) = N(T)^{\perp}$ as desired.
		\end{proof}
	\end{enumerate}
\end{Exercise}