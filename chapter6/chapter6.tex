\setcounter{chapter}{5}
\chapter{Inner Product Spaces}
\thispagestyle{empty}
\newpage

% === Section 6.1 ===
\section{Inner Products and Norms}

% === Exercise 6.1.1 ===
\begin{Exercise}
\begin{enumerate}[(a)]
\item[(a)]
\begin{answer}
False.
\end{answer}
\begin{solution}
It follows from definition immediately.
\end{solution}

\item[(b)]
\begin{answer}
True.
\end{answer}
\begin{solution}
Since the field $F$ means $\mathbb{R}$ or $\mathbb{C}$.
\end{solution}

\item[(c)]
\begin{answer}
False.
\end{answer}
\begin{solution}
It is conjugate linear in the second component by Theorem 6.1.
\end{solution}

\item[(d)]
\begin{answer}
False.
\end{answer}
\begin{solution}
An inner product can be defined by our own. e.g., $\langle u,v\rangle := u v$.
\end{solution}

\item[(e)]
\begin{answer}
False.
\end{answer}
\begin{solution}
It follows from Theorem 6.2.
\end{solution}

\item[(f)]
\begin{answer}
False.
\end{answer}
\begin{solution}
It follows from definition immediately.
\end{solution}

\item[(g)]
\begin{answer}
False.
\end{answer}
\begin{solution}
Let $x = (1,1)$, $y = (1, 0)$, $z=(0,1)$ in $\mathbb{R}^2$. Then $\langle x, y \rangle = \langle x,z \rangle = 1$. However, $y\neq z$.
\end{solution}

\item[(h)]
\begin{answer}
True.
\end{answer}
\begin{solution}
Suppose to contrary that $y\neq 0$. By Theorem 6.1, $\langle x, y \rangle = 0$ implies $x = 0$ or $x = y$. Since $\langle x,y \rangle = 0$ for all $x$, this contradicts. Hence $y = 0$.
\end{solution}

\end{enumerate}
\end{Exercise}

\vspace{12pt}

\setcounter{Exercise}{3}
% === Exercise 6.1.4 ===
\begin{Exercise}
	\begin{enumerate}[(a)]
		\item
		\begin{proof}
			Let $A,B,C\in V$ and $c\in F$. We prove it is compatible with definitions.
			\begin{itemize}
				\item $\langle A+B,C \rangle = \langle A,C \rangle + \langle B,C \rangle$.
				$$
				\langle A+B,C \rangle
				= \trace(C^*(A+B))
				= \trace(C^* A) + \trace(C^* B)
				= \langle A,C \rangle + \langle B,C \rangle.
				$$
				
				\item $\langle c A, B \rangle = c\langle A,B \rangle$.
				$$
				\langle c A,B \rangle
				= \trace(B^*(c A) )
				= c \trace( B^* A)
				= c\langle A,B \rangle.
				$$
				
				\item $\overline{\langle A,B \rangle} = \langle B,A \rangle$.
				$$
				\overline{\langle A,B \rangle}
				= \trace(\overline{B^* A} )
				= \trace(A^* B)
				= \langle B, A \rangle.
				$$
				
				\item $\langle A,A \rangle > 0\iff A\neq O$.
				
				\begin{align*}
				\langle A,A \rangle
				&= tr(A^* A) \\
				&= \sum_{i=1}^{n} \sum_{k=1}^{n} (A^*)_{i k} A_{k i} \\
				&=  \sum_{i=1}^{n} \sum_{k=1}^{n} \overline{A_{k i}} A_{k i} \\
				&= \sum_{i=1}^{n} \sum_{k=1}^{n} |A_{k i}|^2.
				\end{align*}
				Hence $A\neq 0 \iff \exists i,k:A_{k i} \neq 0 \iff \langle A,A \rangle > 0$.
			\end{itemize}
		\end{proof}
	\end{enumerate}
\end{Exercise}

\vspace{12pt}

\setcounter{Exercise}{7}
% === Exercise 6.1.8 ===
\begin{Exercise}
\begin{enumerate}[(a)]
\item
\begin{proof}
Consider $(1,1)\in\mathbb{R}^2$ where $(1,1)$ is nonzero. However, $\langle (1,1),(1,1) \rangle = 1-1 = 0$. This violates the definition. 
\end{proof}
\end{enumerate}
\end{Exercise}

\vspace{12pt}

% === Exercise 6.1.9 ===
\begin{Exercise}
\begin{enumerate}[(a)]
\item
\begin{proof}
Let $\beta = \{z_1,z_2,\cdots, z_n\}$, then we know $\langle x, z_i \rangle = 0$ for each $i$. Since $z\in \beta$, we have $z = \sum_{i=1}^{n} a_i z_i$ for $a_i \in F$. Consider
$$
\langle x,x \rangle
= \langle x, \sum_{i=1}^{n} a_i z_i \rangle
= \sum_{i=1}^{n} \overline{a_i} \langle x, z_i \rangle
= 0.
$$
Hence $x=0$.
\end{proof}

\item
\begin{proof}
Notice that $\langle x,z \rangle = \langle y,z \rangle$ implies $\langle x-y, z \rangle = 0$. Let $\beta = \{z_1,z_2,\cdots, z_n\}$, then we know $\langle x-y, z_i \rangle = 0$ for each $i$. From part (a), we conclude $x-y=0$ implies $x=y$.
\end{proof}
\end{enumerate}
\end{Exercise}

\vspace{12pt}

\setcounter{Exercise}{10}
% === Exercise 6.1.11 ===
\begin{Exercise}
\begin{proof}
Consider for all $x,y\in V$,
\begin{align*}
\| x+y \|^2 + \| x-y \|^2
&= \sqrt{\langle x+y, x+y \rangle}^2 + \sqrt{\langle x-y, x-y \rangle}^2 \\
&= \langle x+y, x+y \rangle + \langle x-y, x-y \rangle \\
&= (\langle x,x+y \rangle + \langle y,x+y \rangle) + (\langle x,x-y \rangle - \langle y,x-y \rangle) \\
&= (\langle x,x+y \rangle + \langle x, x-y \rangle) + (\langle y,x+y \rangle + \langle y,x-y \rangle) \\
&= \langle x,2x \rangle + \langle y,2y \rangle \\
&= 2\langle x,x \rangle + 2\langle y,y \rangle \\
&= 2\|x\|^2 + 2\|y\|^2.
\end{align*}
This means that the sum of square of the four edges is
the sum of square of the two diagonals in the same parallelogram.
\end{proof}
\end{Exercise}

\vspace{12pt}

% === Exercise 6.1.12 ===
\begin{Exercise}
	\begin{proof}
		Since $\{v_1, v_2, \cdots, v_k\}$ is orthogonal, then $\langle v_i, v_j \rangle = 0$ for all $i\neq j$. Follow the hypothesis, then we fix $i$ to consider 
		$$
		\left\langle v_i, \sum_{j=1}^{k} a_j v_j \right\rangle
		= \sum_{j=1}^{k} \overline{a_j} \left\langle v_i, v_j \right\rangle
		= \overline{a_i}\left\langle v_i, v_i \right\rangle.
		$$
		It follows that
		\begin{align*}
		\left\| \sum_{i=1}^{k} a_i v_i \right\|^2
		&= \left\langle \sum_{i=1}^{k} a_i v_i, \sum_{j=1}^{k} a_j v_j \right\rangle \\
		&= \sum_{i=1}^{k} a_i \left\langle v_i, \sum_{j=1}^{k} a_j v_j \right\rangle \\
		&= \sum_{i=1}^{k} a_i \sum_{j=1}^{k} \overline{a_j} \left\langle v_i, v_j \right\rangle \\
		&= \sum_{i=1}^{k} a_i \overline{a_i} \left\langle v_i, v_i \right\rangle \\
		&= \sum_{i=1}^{k} |a_i|^2 \| v_i \|^2.
		\end{align*}
		We obtain the desired result.
	\end{proof}
\end{Exercise}


% === Section 6.2 ===
\section{The Gram-Schmidt Orthogonalization Process and Orthogonal Complements}

% === Exercise 6.2.1 ===
\begin{Exercise}
\begin{enumerate}[(a)]
\item[(a)]
\begin{answer}
False.
\end{answer}
\begin{solution}
Sets of vectors must be linearly independent.
\end{solution}

\item[(b)]
\begin{answer}
True.
\end{answer}
\begin{solution}
It follows from Theorem 6.5.
\end{solution}

\item[(c)]
\begin{answer}
True.
\end{answer}
\begin{solution}
Let $V$ be an arbitrary inner product space and $W$ be a subspace of $V$. Then let $x,y\in W^{\perp}$ and $c$ be a scalar. For all $w\in W$,
$$
\langle x+y, w \rangle = \langle x,w \rangle + \langle y,w \rangle = 0 + 0 = 0;
$$
also
$$
\langle c x, w \rangle = c \langle x,w \rangle = c\cdot 0 = 0.
$$
And $\langle 0, w \rangle = 0$ trivially.

Hence $W^{\perp}$ is a subspace of $V$.
\end{solution}

\item[(d)]
\begin{answer}
False.
\end{answer}
\begin{solution}
The basis must be orthonormal.
\end{solution}

\item[(e)]
\begin{answer}
True.
\end{answer}
\begin{solution}
It follows from definition immediately.
\end{solution}

\item[(f)]
\begin{answer}
False.
\end{answer}
\begin{solution}
An orthogonal set must be nonzero.
\end{solution}

\item[(g)]
\begin{answer}
True.
\end{answer}
\begin{solution}
It follows from the Corollary 2 of Theorem 6.3.
\end{solution}

\end{enumerate}
\end{Exercise}

\vspace{12pt}

% === Exercise 6.2.2 ===
\begin{Exercise}
\begin{enumerate}[(a)]
\item[(b)]
\begin{solution}
First, we try to find an orthogonal basis for $\spann(S)$. Pick $w_1 = (1,1,1)$, $w_2 = (0,1,1)$, $w_3 = (0,0,1)$. Then 
\begin{align*}
v_1 &= w_1 = (1,1,1); \\
v_2 &= w_2 - \frac{\langle w_2, v_1 \rangle}{\|v_1\|^2}v_1
= \frac{1}{3}(-2,1,1); \\
v_3 &= w_3 - \frac{\langle w_3,v_1 \rangle}{\|v_1\|^2}v_1 - \frac{\langle w_3, v_2 \rangle}{\|v_2\|^2}v_2
= \frac{1}{2}(0,-1,1).
\end{align*}
Hence we let $\{v_1, v_2, v_3\}$ is an orthogonal basis for $\spann(S)$, then we normalize them.
\begin{align*}
u_1 &= \frac{1}{\|v_1\|}v_1 = \frac{\sqrt{3}}{3}(1,1,1); \\
u_2 &= \frac{1}{\|v_2\|}v_2 = \frac{\sqrt{6}}{6}(-2,1,1); \\
u_3 &= \frac{1}{\|v_3\|}v_3 = \frac{\sqrt{2}}{2}(0,-1,1).
\end{align*}
Hence $\beta = \{u_1,u_2,u_3\}$ is an orthonormal basis for $\spann(S)$.

Secondly, we find the Fourier coefficients relative to $\beta$.
\begin{align*}
c_1 &= \langle x, u_1 \rangle = \frac{2\sqrt{3}}{3}; \\
c_2 &= \langle x, u_2 \rangle = -\frac{\sqrt{6}}{6}; \\
c_3 &= \langle x, u_3 \rangle = \frac{\sqrt{2}}{2}.
\end{align*}

Thirdly, we verify Theorem 6.5.
$$
x = \langle x, u_1 \rangle u_1 + \langle x,u_2 \rangle u_2 + \langle x, u_3 \rangle u_3.
$$
\end{solution}
\end{enumerate}
\end{Exercise}

\vspace{12pt}

\setcounter{Exercise}{5}
% === Exercise 6.2.6 ===
\begin{Exercise}
\begin{proof}
Let $u\in W$, $y\in W^{\perp}$, then by Theorem 6.6, $x = u+y\in V$ is unique. Since $x\notin W$, then $y\neq 0$. Notice that $\langle u,y \rangle = 0$. Consider
$$
\langle x,y \rangle
= \langle u+y, y \rangle
= \langle u,y \rangle + \langle y,y \rangle
= \langle y,y \rangle
> 0.
$$
Hence $\langle x,y \rangle \neq 0$ follows.
\end{proof}
\end{Exercise}

\vspace{12pt}

\setcounter{Exercise}{12}
% === Exercise 6.2.13 ===
\begin{Exercise}
\begin{enumerate}[(a)]
\item
\begin{proof}
Let $x\in S^{\perp}$. Then $\langle x,y \rangle$ for all $y\in S$. Since $S_0 \subseteq S$, $\langle x,y \rangle$ for all $y\in S_0$. Hence $x\in S_0^{\perp}$. Because $x$ was arbitrary, it follows that $S^{\perp}\subseteq S_0^{\perp}$.
\end{proof}

\item
\begin{proof}
Let $x\in S$, then for any $y\in S^{\perp}$, we have $\langle x,y \rangle = 0$. This means $x\in (S^{\perp})^{\perp}$. Since $x$ was arbitrary, $S\subseteq (S^{\perp})^{\perp}$. Because $\spann(S)$ is the smallest subspace containing $S$, we conclude $\spann(S)\subseteq (S^{\perp})^{\perp}$.
\end{proof}

\item
\begin{proof}
To prove $W = (W^{\perp})^{\perp}$, we need to prove $W \subseteq (W^{\perp})^{\perp}$ and $(W^{\perp})^{\perp} \subseteq W$.

From part (b), we know $W \subseteq (W^{\perp})^{\perp}$. On the other hand, we suppose to contrary that  $(W^{\perp})^{\perp} \nsubseteq W$. Let $x\in (W^{\perp})^{\perp}$, then $x\notin W$. By Exercise 6.2.6, there exists $y\in V$ such that $y\in W^{\perp}$ implies $\langle x,y \rangle \neq 0$. However this contradicts $x\in (W^{\perp})^{\perp}$. Hence $(W^{\perp})^{\perp} \subseteq W$.
\end{proof}

\item
\begin{proof}
To prove $V = W \oplus W^{\perp}$, it suffice to prove $V = W + W^{\perp}$ and $W\cap W^{\perp} = \{0\}$.

By Theorem 6.6, for any $x\in V$, we can find unique $y\in W$ and $z\in W^{\perp}$ such that $x = y+z$. This means $V = W + W^{\perp}$. Let $w\in W\cap W^{\perp}$, then $\langle w, w \rangle=0$. By definition, we know $w = 0$. So $W\cap W^{\perp} = \{0\}$.
\end{proof}

\end{enumerate}
\end{Exercise}

\vspace{12pt}

\setcounter{Exercise}{18}
% === Exercise 6.2.19 ===
\begin{Exercise}
\begin{enumerate}[(a)]
\item[(b)]
\begin{answer}
$\frac{1}{14}(29,17,40)$.
\end{answer}
\begin{solution}
We pick $\{(-3,1,0), (2,0,1)\}$ as a basis for $W$. Pick $w_1 = (-3,1,0)$ and $w_2 = (2,0,1)$. Then use Gram-Schmidt process to compute
\begin{align*}
v_1 &= w_1 = (-3,1,0); \\
v_2 &= w_2-\frac{\langle w_2, v_1 \rangle}{\|v_1\|^2} v_1 = \frac{1}{5}(1,3,5).
\end{align*}
And then normalize them to obtain
\begin{align*}
u_1 &= \frac{v_1}{\|v_1\|} = \frac{1}{\sqrt{10}}(-3,1,0); \\
u_2 &= \frac{v_2}{\|v_2\|} = \frac{1}{\sqrt{35}}(1,3,5).
\end{align*}
So we can find the projection $p$ of $u$ on $W$ is
$$
p = \langle u,u_1\rangle u_1 + \langle u,u_2\rangle u_2
= \frac{1}{14}(29,17,40).
$$
\end{solution}
\end{enumerate}
\end{Exercise}


% === Section 6.3 ===
\section{The Adjoint of a Linear Operator}

% === Exercise 6.3.1 ===
\begin{Exercise}
	\begin{enumerate}[(a)]
		\item[(a)]
		\begin{answer}
			True.
		\end{answer}
		\begin{solution}
			It follows from Theorem 6.9.
		\end{solution}
		
		\item[(b)]
		\begin{answer}
			False.
		\end{answer}
		\begin{solution}
			The form is mapping $V$ to $F$. This violates the definition of linear operator.
		\end{solution}
		
		\item[(c)]
		\begin{answer}
			False.
		\end{answer}
		\begin{solution}
			It violate the hypothesis of Theorem 6.10. $\beta$ must be an orthonormal basis.
		\end{solution}
		
		\item[(d)]
		\begin{answer}
			True.
		\end{answer}
		\begin{solution}
			It follows from Theorem 6.9.
		\end{solution}
		
		\item[(e)]
		\begin{answer}
			False.
		\end{answer}
		\begin{solution}
			Under the hypothesis, it should be $(a T + b U)^* = \overline{a}T^* + \overline{b}U^*$.
		\end{solution}
		
		\item[(f)]
		\begin{answer}
			True.
		\end{answer}
		\begin{solution}
			It follows from the Corollary of Theorem 6.10.
		\end{solution}
		
		\item[(g)]
		\begin{answer}
			True.
		\end{answer}
		\begin{solution}
			It follows from Theorem 6.11.
		\end{solution}
		
	\end{enumerate}
\end{Exercise}

\vspace{12pt}

% === Exercise 6.3.2 ===
\begin{Exercise}
\begin{enumerate}[(a)]
\item[(b)]
\begin{answer}
$y = (1,-2)$.
\end{answer}
\begin{solution}
Since $g(z_1,z_2) = z_1-2z_2 = \langle (z_1,z_2),(1,-2) \rangle$, then we pick $y = (1,-2)$.
\end{solution}
\end{enumerate}
\end{Exercise}

\vspace{12pt}

\setcounter{Exercise}{6}
% === Exercise 6.3.7 ===
\begin{Exercise}
\begin{solution}
We pick $T:\mathbb{R}^2\to\mathbb{R}^2$ such that $T = \begin{pmatrix}
1 & 0 \\
1 & 0
\end{pmatrix}$. Then we know $T^* = \begin{pmatrix}
1 & 1 \\
0 & 0
\end{pmatrix}$. Observe $N(T) = \spann\{(0,1)\}$, $N(T^*) = \spann\{(1,-1)\}$. Notice that $(0,1) \in N(T)$; however $(0,1)\notin N(T^*)$. Hence $N(T)\neq N(T^*)$.
\end{solution}
\end{Exercise}

\vspace{12pt}

% === Exercise 6.3.8 ===
\begin{Exercise}
	\begin{proof}
		Under the hypothesis, we consider
		\begin{align*}
		T^* (T^{-1})^* &= (T^{-1} T)^* = I^* = I; \\
		(T^{-1})^* T^* &= (T T^{-1})^* = I^* = I.
		\end{align*}
		Hence $(T^*)^{-1} = (T^{-1})^*$.
	\end{proof}
\end{Exercise}

\vspace{12pt}

\setcounter{Exercise}{11}
% === Exercise 6.3.12 ===
\begin{Exercise}
	\begin{enumerate}[(a)]
		\item
		\begin{proof}
			It suffice to prove $R(T^*)^{\perp} \subseteq N(T)$ and $N(T) \subseteq R(T^*)^{\perp}$.
			
			First, we claim $N(T) \subseteq R(T^*)^{\perp}$. Let $x\in N(T)$, then $T x = 0$. For any $y\in V$, we consider $
			0 = \langle 0, y \rangle = \langle T x , y \rangle = \langle x, T^* y \rangle$. Hence $x\in R(T^*)^{\perp}$. Because $x$ was arbitrary, the claim holds.
			
			Secondly, we claim $R(T^*)^{\perp} \subseteq N(T)$. Let $x\in R(T^*)^{\perp}$, then for any $y\in V$, we consider $0 = \langle x, T^* y \rangle = \langle T x, y \rangle$. This implies $T x = 0$ and hence $x\in N(T)$. Because $x$ was arbitrary, the claim holds.
			
			Finally, we conclude $R(T^*)^{\perp} = N(T)$.
		\end{proof}
		
		\item
		\begin{proof}
			Since $V$ is finite-dimensional, then both $R(T^*)$ and $N(T)$ are finite-dimensional. From part (a), $(R(T^*)^{\perp})^{\perp} = N(T)^{\perp}$ follows. Hence we obtain $R(T^*) = N(T)^{\perp}$ as desired.
		\end{proof}
	\end{enumerate}
\end{Exercise}


% === Section 6.4 ===
\section{Normal and Self-Adjoint Operators}

% === Exercise 6.4.1 ===
\begin{Exercise}
	\begin{enumerate}[(a)]
		\item[(a)]
		\begin{answer}
			True.
		\end{answer}
		\begin{solution}
			Since $T = T^*$, then $T T^* = T^2 = T^* T$. This means $T$ is normal.
		\end{solution}
		
		\item[(b)]
		\begin{answer}
			False.
		\end{answer}
		\begin{solution}
			Consider $T = \begin{pmatrix}
			3 & 4 \\
			1 & 0
			\end{pmatrix}$, this implies $T$ have eigenvectors $\begin{pmatrix}
			4 \\
			1
			\end{pmatrix}$, $\begin{pmatrix}
			1 \\
			-1
			\end{pmatrix}$. On the other hand, $T^* = \begin{pmatrix}
			3 & 1 \\
			4 & 0
			\end{pmatrix}$. This implies $T^*$ have eigenvectors $\begin{pmatrix}
			1 \\
			1
			\end{pmatrix}$, $\begin{pmatrix}
			-1 \\
			4
			\end{pmatrix}$. Hence their eigenvectors are different.
		\end{solution}
		
		\item[(c)]
		\begin{answer}
			False.
		\end{answer}
		\begin{solution}
			$\beta$ should be an orthonormal basis. It follows from the definition of normal operator and Theorem 6.10. A counter-example could see Exercise 6.4.3.
		\end{solution}
		
		\item[(d)]
		\begin{answer}
			True.
		\end{answer}
		\begin{solution}
			It follows from Theorem 6.10.
		\end{solution}
		
		\item[(e)]
		\begin{answer}
			True.
		\end{answer}
		\begin{solution}
			It follows from the Lemma in page 373.
		\end{solution}
		
		\item[(f)]
		\begin{answer}
			True.
		\end{answer}
		\begin{solution}
			Since $I^* = I$ and $O^* = O$.
		\end{solution}
		
		\item[(g)]
		\begin{answer}
			False.
		\end{answer}
		\begin{solution}
			Consider $T = \begin{pmatrix}
			0 & -1 \\
			1 & 0
			\end{pmatrix}$, then $T$ is a normal operator. However, $T$ is not diagonalizable.
		\end{solution}
		
		\item[(h)]
		\begin{answer}
			True.
		\end{answer}
		\begin{solution}
			If the inner product space is over $\mathbb{R}$, then it follows from Theorem 6.17; if it is over $\mathbb{C}$, it follows from Theorem 6.16.
		\end{solution}
		
	\end{enumerate}
\end{Exercise}

\vspace{12pt}

% === Exercise 6.4.2 ===
\begin{Exercise}
\begin{enumerate}[(a)]
\item[(a)]
\begin{solution}
Observe $T = \begin{pmatrix}
2 & -2 \\
-2 & 5
\end{pmatrix} = T^*$. Hence $T$ is self-adjoint and normal. Solve $\det(T - \lambda I) = \lambda^2-7\lambda+6 = 0$ to get $\lambda_1 = 6$, $\lambda_2 = -1$. Then we have eigenvector $(1,-2)$ corresponding to $\lambda_1$, and eigenvector $(2,1)$ corresponding to $\lambda_2$. And normalize them to obtain $$
\left\{\frac{1}{\sqrt{3}}(2,1),\frac{1}{\sqrt{3}}(1,-2)\right\}
$$
is an orthonormal basis for $V$.
\end{solution}
\end{enumerate}
\end{Exercise}

\vspace{12pt}

% === Exercise 6.4.3 ===
\begin{Exercise}
	\begin{answer}
		$T(x,y) = (x, 2y)$ and $\{(1,1),(0,1)\}$ is a basis for $\mathbb{R}^2$.
	\end{answer}
	\begin{solution}
		Pick $T:\mathbb{R}^2\to \mathbb{R}^2$ where $T(x,y) = (x,2y)$. Then $T = \begin{pmatrix}
		1 & 0 \\
		0 & 2 
		\end{pmatrix}$. Observe $T = T^*$ and hence $T$ is normal.
		
		Now pick $\beta = \{(1,1),(0,1)\}$, then
		$$
		[T]_{\beta} = \begin{pmatrix}
		1 & 0 \\
		1 & 1
		\end{pmatrix} T = \begin{pmatrix}
		1 & 0 \\
		1 & 2
		\end{pmatrix}.
		$$
		Consequetly, we verify whether it is normal or not.
		\begin{align*}
		[T]_{\beta} [T]_{\beta}^* &= \begin{pmatrix}
		1 & 0 \\
		1 & 2
		\end{pmatrix} \begin{pmatrix}
		1 & 1 \\
		0 & 2
		\end{pmatrix} = \begin{pmatrix}
		1 & 1 \\
		1 & 5
		\end{pmatrix}; \\
		[T]_{\beta}^* [T]_{\beta} &= \begin{pmatrix}
		1 & 1 \\
		0 & 2
		\end{pmatrix} \begin{pmatrix}
		1 & 0 \\
		1 & 2
		\end{pmatrix} = \begin{pmatrix}
		2 & 2 \\
		2 & 4
		\end{pmatrix}.
		\end{align*}
		Hence $[T]_{\beta}$ is not normal.
	\end{solution}
\end{Exercise}

\vspace{12pt}

\setcounter{Exercise}{5}
% === Exercise 6.4.6 ===
\begin{Exercise}
	\begin{enumerate}[(a)]
		\item
		\begin{proof}
			Consider
			$$
			T_1^*
			= \left[ \frac{1}{2}(T+T^*) \right]^*
			= \frac{1}{2} (T+T^*)^*
			= \frac{1}{2} (T^*+T)
			= \frac{1}{2} (T + T^*)
			= T_1.
			$$
			This means $T_1$ is self-adjoint.
			
			Consider
			$$
			T_2^*
			= \left[ \frac{1}{2}\ii(T-T^*) \right]^*
			= \frac{1}{-2\ii}(T-T^*)^*
			= -\frac{1}{2\ii}(T^*-T)
			= \frac{1}{2\ii}(T-T^*)
			= T_2.
			$$
			This means $T_2$ is self-adjoint.
			
			Consider
			$$
			T_1 + \ii T_2
			= \frac{1}{2}(T+T^*) + \ii \cdot \frac{1}{2\ii} (T-T^*)
			= \frac{1}{2}T+ \frac{1}{2}T^* + \frac{1}{2}T - \frac{1}{2}T^*
			= T.
			$$
			We obtain the desired result.
		\end{proof}
		
		\item
		\begin{proof}
			Suppose $T = U_1+\ii U_2$ with $U_1 = U_1^*$ and $U_2 = U_2^*$.
			
			Consider
			$$
			T_1
			= \frac{1}{2}(T+T^*)
			= \frac{1}{2}(U_1 + \ii U_2 + U_1^* - \ii U_2^*)
			= \frac{1}{2}(U_1 + \ii U_2^* + U_1 - \ii U_2^*)
			= U_1.
			$$
			On the other hand,
			$$
			T_2
			= \frac{1}{2\ii}(T-T^*)
			= \frac{1}{2\ii}(U_1 + \ii U_2 - U_1^* + \ii U_2^*)
			= \frac{1}{2\ii}(U_1^*+\ii U_2 - U_1^* + \ii U_2)
			= U_2.
			$$
			
			As a result, we obtain the desired results.
		\end{proof}
		
		\item
		\begin{proof}
			$(\Longrightarrow)$
			Since $T$ is normal, then $T T^* = T^*T$. Consider
			\begin{align*}
			T_1 T_2 
			&= \frac{1}{2}(T+T^*)\cdot \frac{1}{2\ii}(T-T^*)
			= \frac{1}{4\ii}(T^2+T^* T-T T^*-T)
			= \frac{1}{4\ii}(T^2-T); \\
			T_2 T_1
			&= \frac{1}{2\ii}(T-T^*)\cdot \frac{1}{2}(T+T^*)
			= \frac{1}{4\ii}(T^2-T^*T+T T^*-T)
			= \frac{1}{4\ii}(T^2-T).
			\end{align*}
			It follows that $T_1 T_2 = T_2 T_1$.
			
			\vspace{2ex}
			
			$(\Longleftarrow)$
			Compute $T_1 T_2$ and $T_2 T_1$, since $T_1 T_2 = T_2 T_1$, we have
			$$
			\frac{1}{4\ii}(T^2+T^* T-T T^*-T)
			= \frac{1}{4\ii}(T^2-T^*T+T T^*-T).
			$$
			This implies
			$$
			2T^* T = 2T T^*.
			$$
			Then $T^* T = T T^*$ follows and hence $T$ is normal.
		\end{proof}
	\end{enumerate}
\end{Exercise}

\vspace{12pt}

\setcounter{Exercise}{8}
% === Exercise 6.4.9 ===
\begin{Exercise}
	\begin{proof}
		Since $T$ is normal, if $\|T x\| = 0$ for arbitrary $x\in V$, then $\|T^* x \| = 0$. This converse also holds. Hence $N(T) = N(T^*)$. By Exercise 6.3.12, we know 
		$$
		R(T^*) = N(T)^{\perp} = N(T^*)^{\perp} = \big(R(T)^{\perp}\big)^{\perp} = R(T).
		$$
	\end{proof}
\end{Exercise}

% === Section 6.5 ===
%\section{Unitary and Orthogonal Operators and Their Matrices}


% === Section 6.6 ===
%\section{Orthogonal Projections and the Spectral Theorem}


% === Section 6.7 ===
%\section{The Singular Value Decomposition and the Pseudoinverse}


% === Section 6.8 ===
%\section{Bilinear and Quadratic Forms}


% === Section 6.9 ===
%\section{Einstein's Special Theory of Relativity}


% === Section 6.10 ===
%\section{Conditioning and the Rayleigh Quotient}


% === Section 6.11 ===
%\section{The Geometry of Orthogonal Operators}