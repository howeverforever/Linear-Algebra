% === Exercise 1.3.31 ===
\begin{Exercise}
	\begin{enumerate}[(a)]
		\item 
		\begin{proof}
			$(\Longrightarrow)$
			Consider
			$$
			0 = v + (-v) \in v+W.
			$$
			Then $-v\in W$ by definition from $v+W$. Since $W$ is a subspace of $V$ over $F$ by hypothesis, then $v\in W$.
			
			\vspace{2ex}
			
			$(\Longleftarrow)$
			Since $v\in W$, then $v+W = W$ is a subspace of $V$ trivially.
		\end{proof}
		
		\item 
		\begin{proof}
			Consider
			\begin{alignat*}{7}
				\quad&& v_1 + W &= v_2 + W \\
				\iff&& v_1-v_2 + W &= W \\
				\iff&& v_1-v_2 &\in W.
			\end{alignat*}
			Hence we complete the proof.
		\end{proof}
		
		\item
		\begin{proof}
			Since $v_1+W=v'_1+W$ and $v_2+W=v'_2+W$, then from part (b), this implies $v_1-v'_1\in W$ and $v_2-v'_2\in W$. Since $W$ is a subspace of $V$ over $F$, then
			$$
			(v_1-v'_1)+(v_2-v'_2)
			= (v_1+v_2)-(v'_1+v'_2)\in W.
			$$
			Now from part (b) again, we have
			$$
			(v_1+v_2)+W = (v'_1+v'_2)+W.
			$$
			By definition of addition in cosets of $W$, we have
			$$
			(v_1+W)+(v_2+W) = (v'_1+W)+(v'_2+W).
			$$
			
			On the other hand, we need to prove $a(v_1+W) = a(v'_1+W)$.
			
			Since $v_1+W$ and $v'_1+W$ are subspaces of $W$, then $v_1,v'_1\in W$ from part (a). Moreover, $a v_1, a v'_1\in W$ since $W$ is a subspace of $V$ over $F$ and $a\in F$. It follows that $a v_1-a v'_1\in W$. From part (b), we know
			\begin{equation}\label{eq:1.3.31}
				a v_1+W = a v'_1+W.
			\end{equation}
			
			Consider for all $a\in F$, then by definition of scalar multiplication by scalars of $F$ and equation \eqref{eq:1.3.31}, we have
			$$
			a(v_1+W)
			= a v_1 + W
			= a v'_1 + W
			= a(v'_1 + W).
			$$
			
			Hence we conclude the two operations are well defined.
		\end{proof}
		
		\item
		\begin{proof}
			Notice that $S=V/W=\left\{v+W:v\in V\right\}$. Now we need to prove $S$ is compatible with eight definitions.
			
			Suppose the zero vector in $S$ is $0+W$. Here we go by verifying patiently.
			
			\begin{itemize}
				\item $\mathbf{(VS\ 1)}$
				For all $v_1+W,v_2+W \in S$, we have
				$$
				(v_1+W)+(v_2+W)
				= (v_1+v_2)+W
				= (v_2+v_1)+W
				= (v_2+W)+(v_1+W).
				$$
				
				\item $\mathbf{(VS\ 2)}$
				For all $v_1+W,v_2+W,v_3+W\in S$, we have
				\begin{align*}
					(v_1+W+v_2+W)+(v_3+W)
					&= (v_1+v_2)+W)+(v_3+W) \\
					&= (v_1+v_2+v_3)+W) \\
					&= (v_1+W) + ((v_2+v_3)+W) \\
					&= (v_1+W) + (v_2+W+v_3+W).
				\end{align*}
				
				\item $\mathbf{(VS\ 3)}$
				For all $v_1+W\in S$, there is a zero vector $0+W$ so that
				$$
				v_1+W+(0+W)
				= (v_1+0)+W
				= v_1+W.
				$$
				
				\item $\mathbf{(VS\ 4)}$
				For all $v_1+W\in S$, we pick $-v_1+w\in S$ since $v_1\in V$ and $-v_1\in V$. So $-v_1+w\in S$. Then we have
				$$
				(v_1+W)+(-v_1+W)
				= (v_1-v_1)+W
				= 0+W.
				$$
				
				\item $\mathbf{(VS\ 5)}$
				For all $v_1+W\in S$, we have
				$$
				1(v_1+W)
				= v_1+W.
				$$
				
				\item $\mathbf{(VS\ 6)}$
				For all $a,b\in F$ and $v_1+W\in S$, we have
				\begin{align*}
					(ab)(v_1+W)
					&= ab v_1 + W \\
					&= a(b v_1+W) \\
					&= a(b(v_1+W)).
				\end{align*}
				
				\item $\mathbf{(VS\ 7)}$
				For all $a\in F$ and $v_1+W,v_2+W\in S$, we have
				\begin{align*}
					a(v_1+W+v_2+W)
					&= a((v_1+v_2)+W) \\
					&= (a v_1 + a v_2) + W \\
					&= (a v_1 + W) + (a v_2 + W)\\
					&= a(v_1 + W)+a(v_2 + W).
				\end{align*}
				
				\item $\mathbf{(VS\ 8)}$
				For all $a,b\in F$ and $v_1+W\in S$, we have
				\begin{align*}
					(a+b)(v_1+W)
					&= ((a+b)v_1) + W \\
					&= (a v_1 + b v_1) + W \\
					&= (a v_1 + W) + (b v_1 + W) \\
					&= a(v_1+W) + b(v_1+W).
				\end{align*}
			\end{itemize}
			Hence $S$ is a vector space with the operations defined in part (c).
		\end{proof}
	\end{enumerate}
\end{Exercise}