% === Exercise 1.3.19 ===
\begin{Exercise}
	\begin{proof}
		$(\Longrightarrow)$
		
		Suppose to contrary that $W_1\nsubseteq W_2$ and $W_2\nsubseteq W_1$. Notice that $W_1\cup W_2$ is a subspace of $V$. Let $x\in W_1\setminus W_2$ and $y\in W_2\setminus W_1$, then $x+y\in W_1$ or $x+y\in W_2$.
		
		If $x+y\in W_1$, then since $y=(x+y)-x\in W_1$, this contradicts that $y\notin W_1$.
		
		Otherwise, if $x+y\in W_2$, then since $x=(x+y)-y\in W_2$, this contradicts that $x\notin W_2$.
		
		In either of two cases, it always leads to a contradiction. Hence $W_1\subseteq W_2$ or $W_2\subseteq W_1$.
		
		\vspace{2ex}
		
		$(\Longleftarrow)$
		
		Since $W_1\subseteq W_2$ or $W_2\subseteq W_1$, then $W_1\cup W_2 = W_1$ or $W_1\cup W_2 = W_2$. In either of two cases, $W_1\cup W_2$ is a subspace of $V$ by hypothesis that $W_1$ and $W_2$ are subspaces of $V$.
	\end{proof}
\end{Exercise}