% === Exercise 5.2.7 ===
\begin{Exercise}
\begin{answer}
$A^n = \frac{1}{3}\begin{pmatrix}
5^n+2(-1)^n & 2\cdot 5^n - 2(-1)^n \\
5^n-(-1)^n & 2\cdot 5^n + (-1)^n
\end{pmatrix}$.
\end{answer}
\begin{solution}
Solve $\det(A-\lambda I) = 0$, we have $(\lambda-5)(\lambda+1) = 0$. Pick $\lambda_1 = 5$, $\lambda_2 = -1$. For $\lambda_1 = 5$, we know $\left\{\begin{pmatrix}
1 \\
1
\end{pmatrix}\right\}$ is a basis for $N_{\lambda_1}$. For $\lambda_2 = -1$, we know $\left\{\begin{pmatrix}
-2 \\
1
\end{pmatrix}\right\}$ is a basis for $N_{\lambda_2}$. Hence we pick $Q = \begin{pmatrix}
1 & -2 \\
1 & 1
\end{pmatrix}$. It follows that
$$
\begin{pmatrix}
5 & 0 \\
0 & -1
\end{pmatrix} = Q^{-1} A Q.
$$
This implies
$$
\begin{pmatrix}
5 & 0 \\
0 & -1
\end{pmatrix}^n = (Q^{-1} A Q)^n = Q^{-1} A^n Q.
$$
So we conclude
\begin{align*}
A^n 
&= Q \begin{pmatrix}
5^n & 0 \\
0 & (-1)^n
\end{pmatrix} Q^{-1} \\ 
&= \begin{pmatrix}
1 & -2 \\
1 & 1
\end{pmatrix} \begin{pmatrix}
5^n & 0 \\
0 & (-1)^n
\end{pmatrix} \left( \frac{1}{3}\begin{pmatrix}
1 & 2 \\
-1 & 1
\end{pmatrix} \right) \\
&= \frac{1}{3}\begin{pmatrix}
5^n+2(-1)^n & 2\cdot 5^n - 2(-1)^n \\
5^n-(-1)^n & 2\cdot 5^n + (-1)^n
\end{pmatrix}.
\end{align*}
\end{solution}
\end{Exercise}