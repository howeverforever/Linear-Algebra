% === Exercise 5.4.1 ===
\begin{Exercise}
\begin{enumerate}[(a)]
\item[(a)]
\begin{answer}
False.
\end{answer}
\begin{solution}
Since $\{0\}$ is subspace of $T$ trivially, and $\{0\}$ is $T$-invariant subspace.
\end{solution}

\item[(b)]
\begin{answer}
True.
\end{answer}
\begin{solution}
It follows from Theorem 5.21.
\end{solution}

\item[(c)]
\begin{answer}
False.
\end{answer}
\begin{solution}
Define $T:\mathbb{R}\to\mathbb{R}$ by $T(x) = x$. Pick $v=\{1\}$, $w=\{-1\}$. This implies $W=W'=\mathbb{R}$. However, $v\neq w$.
\end{solution}

\item[(d)]
\begin{answer}
False.
\end{answer}
\begin{solution}
Define $T:\mathbb{R}^2\to\mathbb{R}^2$ by $T(x,y) = x$. Pick $v = (1,1)$, then $T$-cyclic subspace generated by $v$ is $\mathbb{R}^2$. However, one generated by $T(v)$ is $y$-axis which in $\mathbb{R}$. Clearly, $\mathbb{R}^2 \neq \mathbb{R}$.
\end{solution}

\item[(e)]
\begin{answer}
True.
\end{answer}
\begin{solution}
It follows from Cayley-Hamilton Theorem (Theorem 5.23). That is, $g(t)$ is the characteristic polynomial of $T$.
\end{solution}

\item[(f)]
\begin{answer}
True.
\end{answer}
\begin{solution}
It follows from Exercise 5.1.29(a).
\end{solution}

\item[(g)]
\begin{answer}
True.
\end{answer}
\begin{solution}
It follow from Theorem 5.25.
\end{solution}

\end{enumerate}
\end{Exercise}