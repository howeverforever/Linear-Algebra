% === Exercise 5.1.1 ===
\begin{Exercise}
	\begin{enumerate}[(a)]
		\item[(a)]
		\begin{answer}
			False.
		\end{answer}
		\begin{solution}
			$I_2$ has duplicate eigenvalues $1$.
		\end{solution}
		
		\item[(b)]
		\begin{answer}
			True.
		\end{answer}
		\begin{solution}
			Since a real matrix $A$ has one eigenvector $v$, by Theorem 5.4, we know $(A-\lambda I)(v) = 0$ for $v\neq 0$. It follows that $(A-\lambda I)(t v) = t(A-\lambda I)(v) = t\cdot 0 = 0$ for $t\in\mathbb{R}$. This means $t v$ is also an eigenvector. Since $\mathbb{R}$ is an infinite set, then we have infinite eigenvectors. 
			
			Notice that this statement will fail if the field is finite.
		\end{solution}
		
		\item[(c)]
		\begin{answer}
			True.
		\end{answer}
		\begin{solution}
			Consider the square matrix $\begin{pmatrix}
			0 & 1 \\
			-1 & 0
			\end{pmatrix}$. Since it has no eigenvalues, and hence has no eigenvectors.
			Note: It also depends on the field in which this matrix is defined. For example, the given matrix above has both eigenvalues and eigenvectors over the complex field.
		\end{solution}
		
		\item[(d)]
		\begin{answer}
			False.
		\end{answer}
		\begin{solution}
			The zero matrix $\begin{pmatrix}
			0 & 0 \\
			0 & 0
			\end{pmatrix}$ has duplicate eigenvalues $0$.
		\end{solution}
		
		\item[(e)]
		\begin{answer}
			False.
		\end{answer}
		\begin{solution}
			The vectors $(1,0),(2,0)$ are eigenvectors of $I_2$. However they are linearly dependent.
		\end{solution}
		
		\item[(f)]
		\begin{answer}
			False.
		\end{answer}
		\begin{solution}
			Consider $I_2$ which has two eigenvalues $1,1$; however $1+1=2$, and $2$ is not an eigenvalue of $I_2$ obviously.
		\end{solution}
		
		\item[(g)]
		\begin{answer}
			False.
		\end{answer}
		\begin{solution}
			Put a linear operator $T$ on $P(\mathbb{R})$. Then $T$ has an eigenvalue $1$.
		\end{solution}
		
		\item[(h)]
		\begin{answer}
			True.
		\end{answer}
		\begin{solution}
			Put $A = Q^{-1} D Q$ where $D$ is a diagonal matrix and $P,Q\in M_{n\times n}(F)$. Let $\alpha$ be a basis and $\beta$ be the standard basis for $F_n$. Since $Q$ is invertible, then $Q = [I]_{\beta}^{\alpha}$. Since $A$ is diagonalizable, by Theorem 5.1, we know $\beta$ can be consisted of the eigenvectors of $A$.
		\end{solution}
		
		\item[(i)]
		\begin{answer}
			True.
		\end{answer}
		\begin{solution}
			Suppose $A$ is similar to $B$, then there exists $Q$ which is invertible such that $A = Q^{-1} B Q$. If $Av = \lambda v$ for any eigenvalues $\lambda$ corresponding to some eigenvectors $v$, then we have
			\begin{alignat*}{7}
			\quad&& Av &= \lambda v \\
			\implies&& (Q^{-1} B Q ) v &= \lambda v \\
			\implies&& (B Q)v &= Q \lambda v \\
			\implies&& B(Q v) &= \lambda (Q v).
			\end{alignat*}
			This means $A,B$ have the same eigenvalues.
		\end{solution}
		
		\item[(j)]
		\begin{answer}
			False.
		\end{answer}
		\begin{solution}
			Suppose $A$ is similar to $B$, then there exists $Q$ which is invertible such that $A = Q^{-1} B Q$. Put $A = \begin{pmatrix}
			1 & 1 \\
			0 & 2
			\end{pmatrix}$ and $B = \begin{pmatrix}
			2 & 0 \\
			1 & 1
			\end{pmatrix}$ and $Q = Q^{-1} = \begin{pmatrix}
			0 & 1 \\
			1 & 0
			\end{pmatrix}$.
			It follows that $(1,0)$ is an eigenvector of $A$, but so is not $B$.
		\end{solution}
		
		\item[(k)]
		\begin{answer}
			False.
		\end{answer}
		\begin{solution}
			Put $T = \begin{pmatrix}
			2 & 1 \\
			3 & 0
			\end{pmatrix}$. Then $(1,1)$ and $(1,-3)$ are eigenvectors of $T$; however, $(1,1)+(1,-3) = (2,-2)$ is not an eigenvector of $T$.
		\end{solution}
		
	\end{enumerate}
\end{Exercise}
