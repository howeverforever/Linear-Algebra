% === Exercise 5.1.11 ===
\begin{Exercise}
\begin{enumerate}[(a)]
\item
\begin{proof}
Since $A$ is similar to $\lambda I$, there exists $Q$ which is invertible such that $A = Q^{-1} \lambda I Q$. It follows that
$$
A = \lambda Q^{-1} I Q = \lambda Q^{-1} Q = \lambda I.
$$
\end{proof}

\item
\begin{proof}
Let $\beta = \{v_1,v_2,\cdots,v_n\}$ is a basis for $A$ and $v_i$ is an eigenvector of $A$ for each $i$. By Theorem 5.1, we know $[A]_{\beta}$ is a diagonal matrix and hence is a upper triangular matrix. From Exercise 5.1.9, and by hypothesis that all eigenvalues are the same called $\lambda$, then $[A]_{\beta} = \lambda I$. This implies $A$ is similar to $\lambda I$, from part (a), we conclude $A = \lambda I$.
\end{proof}

\item
\begin{proof}
Let $A = \begin{pmatrix}
1 & 1 \\
0 & 1
\end{pmatrix}$. Since $A$ has the same eigenvalues $1$, from part (b), if $A$ is diagonalizable, then $A$ must be a scalar matrix; however, $A$ is not a scalar matrix and hence is not diagonalizable. 
\end{proof}
\end{enumerate}
\end{Exercise}